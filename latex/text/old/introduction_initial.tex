
\chapter{Introduction}

\section*{Virtual reality}
Virtual reality (VR) is an umbrella term for a range of advanced human-computer interface technologies aimed at 
creating the illusion of being in a virtual environment, as well as using natural human motion to interact with said environment.
These technologies provide a multitude of creative and commercial opportunities, 
so research as well as adoption have gained significant traction over the recent years. 

The most prominent of these technologies are virtual reality goggles or glasses.
These head-mounted displays handle the visual part of creating an illusion of being in a virtual space.
This is achieved thanks to two main differences with traditional displays.
Being so close to the eyes allows the glasses' display(s) to cover most of a person's field of vision..
The illusion of depth is achieved by displaying a slightly different image to each eye.
Most virtual reality glasses also support head movement tracking (commonly simply referred to as ``head tracking"),
although external sensors are often required to make it work.
Head tracking allows the user to look around in the virtual space, further improving immersion.

Other technologies pursuing the same objective of maximizing immersion in a virtual space exist, and new ones are being developed at a rapid pace.
However most of them, with the notable exception of three dimensional audio, are going to be irrelevant for the purposes of this thesis.

\section*{360\degree{} video}

360\degree{} video (also referred to as ``360 video" or ``360\degree{} video") utilises virtual reality technologies to create an immersive but usually noninteractive viewing experience.
Like other types of VR content, 360\degree{} video benefits from the ability to look around and perceive depth,
but shares more similarity with traditional video content such as movies and TV shows,
therefore underutilising VR's potential in comparison to virtual reality games and other highly interactive experiences. 
This approach however allows to use content creation workflows more similar to the ones already well established in traditional (nonimmersive) video creation.

\section*{The role of sound in immersive experiences}

Until this point I have consciously avoided mentioning a vital part of fooling the human brain into thinking 
it is in a different environment to the one it is actually in - audio.
Although it's importance is easily outshined by the importance of sight, hearing is an essential part of human perception, 
and getting audio wrong in a VR context easily ruins the whole experience.
While simulating the perception of touch - another vital sense - is a very difficult task, doing the same for hearing is fortunately already within our reach.
To do so, as with video, the extension of audio into the third dimension is required.

Similar to depth perception, the position of sound sources in space is extrapolated [REPLACE?] by our brain using the differences in
sound perceived by the left and right ears.
Binaural audio (which will be described in more detail as part of this thesis) utilises this fact to allow the listener to
perceive the spatial position of sounds with great accuracy. 
Furthermore, it only requires a pair of headphones, which makes it perfect for VR applications.

\section*{The aims of this thesis}

Until this point I have consciously avoided mentioning a vital part of fooling the human brain into thinking 
it is in a different environment to the one it is actually in - audio.
Although it's importance is easily outshined by the importance of sight, hearing is an essential part of human perception, 
and getting audio wrong in a VR context easily ruins the whole experience.
While simulating the perception of touch - another vital sense - is a very difficult task, doing the same for hearing is fortunately already within our reach.
To do so, as with video, the extension of audio into the third dimension is required.

Similar to depth perception, the position of sound sources in space is extrapolated [REPLACE?] by our brain using the differences in
sound perceived by the left and right ears.
Binaural audio (which will be described in more detail as part of this thesis) utilises this fact to allow the listener to
perceive the spatial position of sounds with great accuracy. 
Furthermore, it only requires a pair of headphones, which makes it perfect for VR applications.